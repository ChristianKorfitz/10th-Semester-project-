

In summary, there is still a need for prosthetic devices to fully close the neural loop, thus providing amputees with proprioceptive feedback to lower the visual attention needed when having to control a myoelectric prosthesis. Using the modality of electrotactile stimulation as a mean of transferring information of the prosthetic state offers multiple stimulation parameters which can be modulated through several channels enabling possibilities for intuitive and meaningful sensory feedback. \\
However, even though several opportunities present themselves in modulating the stimulation amplitude, frequency and active channels, it would be of great interest to investigate which would lead the sensory feedback to be perceived most intuitively. As presented in \secref{Maxxx} the frequency cannot be controlled individually for each pad in the electrode, thus a feedback scheme modulating frequency will no be investigated in this study. This leaves the following study objective: 

\begin{center}
	\textit{Test and evaluate two novel stimulation schemes, one based on modulating amplitude and one based on spatial localization of activation, for conveying sensory feedback of the prosthetic state in a closed loop prosthetic control system.}  
\end{center} 