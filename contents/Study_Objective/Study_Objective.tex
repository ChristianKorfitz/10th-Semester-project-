

In summary, there is still a need for myoelectric prosthetic devices to fully close the neural loop by providing amputees with proprioceptive feedback to lower the need for visual attention. As presented in \secref{SoA} most studies have focused on providing exteroceptive feedback, while only very few studies have investigated how proprioceptive information could be conveyed to aid prosthetic control in cases where visual attention is less wanted. Using the modality of electrotactile stimulation as a mean of transferring information of position states offers multiple stimulation parameters which can be modulated through several channels enabling possibilities for intuitive and meaningful sensory feedback. However, even though several opportunities present themselves in modulating the stimulation amplitude, frequency and active channels, it would be of great interest to investigate which modulation would lead the sensory feedback to be perceived most intuitively. As stated in \secref{Maxxx} the frequency cannot be controlled individually for each pad in the electrode, thus a feedback scheme modulating frequency will not be investigated in this study. 

Investigating whether spatially coded or amplitude coded information assists control the most when neglecting visual attention, will provide insight into which parameters future configurations should encapsulate. This leaves the following study objective: 

\begin{center}
	\textit{Test and evaluate two novel stimulation schemes, one based on modulating amplitude and one based on spatial localization of activation, for conveying sensory feedback of the position state in a closed-loop prosthetic control system.}  
\end{center} 