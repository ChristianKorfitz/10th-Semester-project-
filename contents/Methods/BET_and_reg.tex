\section{Brain Extraction and Registration} \label{BET}

In order to study activation in the brain, the brain has to be segmented and extracted from the surrounding tissue. To achieve an image only containing the brain, there was made use of the FSL Brain Extraction Tool (BET).  BET was used for extracting the brain in both the structural T$_1$-weighted image and in the functional T$_{2}^*$-weighted image sequence. As this is a standard applied method and is done before the implementation of preprocessing methods, it is seen non essential for the focus of this project. Hence, no further documentation of this step will be presented. For further documentation of BET specifications we refer to Smith et al. \cite{Smith2002}.    \\
The next step of preparing the functional images for analysis was to obtain the same coordinate system for these and the structural images. This step is called registration and is also known as spatial normalization. There are two levels of registration: intra-subject and inter-subject. Intra-subject is registering different scan sequences within the same subject and is done to spatially localize the activation. As the functional images contain very low resolution the localization of activation it can be hard to assess and therefore activation localization can more easily be studied by registering the functional image with a structural image. Inter-subject is to obtain the same brain localization in the coordinate system, called common space, for all participants. This is done to achieve population analysis, which facilitates result to be interpreted and reported objectively and consistently across studies. Inter-subject registration can be fulfilled by registering the brain for each participant to the often used Montreal Neurological Institute (MNI) template. \cite{Hajnal2001} \\
In this project a two step registration was implemented. First, a intra-subject registration was completed by registering the functional to the structural by utilizing the FSL FMRIB'S Linear Image Registration Tool (FLIRT), which uses affine linear transformation. \cite{Jenkinson2001} Secondly, a inter-subject registration was implemented, registering the structural for all participants to the MNI template. This was achieved by implementing both FLIRT and FMRIB'S Non-linear Image Regristration Tool (FNIRT), which uses non linear transformation. \cite{Andersson2007}   

