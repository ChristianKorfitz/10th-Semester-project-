\section{Data Set Structuring}

As presented in the previous section the number of participants included in this project was 139. Each participant underwent 3 heat runs where, for each, a functional scan was acquired, summed to a total of 417 scans. In \secref{art} it was introduced that the FIX preprocessing method utilizes a classification algorithm for separating noise sources from signal sources. The FIX tool package comes with predefined training sets for training the classifier. However, none of these have been made from fMRI scans consisting of signal related to the pain response from a noxious stimulus block design. Therefore, it was chosen to split the total data set into a training data set for training the classifier for this specific application and a test data set for evaluating the performance of both preprocessing methods. \\
Approximately one fourth of the entire data set was to be used as training data. In Salimi-Khorshidi et al. \cite{Salimi-Khorshidi2014} it was stated that a minimum of 10 participants were required to train the classifier. With consideration to the hypothesized higher complexity of the acquired data, it was chosen to use data from a larger amount of participants for training the classifier. However, it was prioritized that the data comprised of more than 100 participants to retain high statistical power.\\
The training data set consisted of 34 participants making a total of 102 scans. However, during the initial analyses two heat scans were found to have missing data resulting in the exclusion of these. This meant that the test data set consisted of the remaining 105 participants. However four participants were found to not having the cerebellum fully scanned, which resulted in additional exclusion of these four participants. In the remaining test data set 7 scans were found to have a missing heat run scan.  Therefore, the test data set consisted of a total of 296 scans.  \\

%
% \fxnote{regarding the inception of the rating periods of the scan, which resulted in the exclusion of these two heat scans. Furthermore, four subjects were found to not having the cerebellum fully scanned, which resulted in additional exclusion of these four subjects. }