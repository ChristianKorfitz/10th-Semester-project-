
\section{The Prosthetic Control System}

Having extracted features from the three EMG datasets of one movement for each of the four movements, the control system could be build in order to achieve real-time recognition of movements. The following sections will explain the implementation of the control system and how the achieved level of control was assessed for each subject. 

\subsection{Building the Control System} 

The implementation of the control system was divided into two parts. To achieve recognition of performed movement a classifier was trained, however this only produces a recognition of a movement and does not reflect the intensity of which the movement is being performed with. Therefore, following the recognition of performed movement a linear regression model was implemented to achieve proportional control. 

\subsubsection{Classification of movement}

Real-time classification of movements was accomplished by implementing a LDA classifier. As presented in \secref{patter} the classifier needed to be trained using data from each movement. Hence, the five features extracted for each of the 40 $\percent$, 50 $\percent$ and 70 $\percent$ fraction of the pMVC for one movement were assembled into one labeled training matrix. The same was done for the three remaining movements. A fifth class was labeled rest and its training matrix only contained the features from the single rest acquisition.  

3 intensities for each movement 
4 movements + 1 rest 
5 features 
8 channels 

\subsubsection{Proportional control}  



\subsection{Assessment of Subject Control}
