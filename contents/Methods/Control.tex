
\section{The Prosthetic Control System}

Having extracted features from the three EMG datasets of one movement for each of the four movements, the control system could be build in order to achieve real-time recognition of movements. The following sections will explain the implementation of the control system and how the achieved level of control was assessed for each subject. 

\subsection{Building the Control System} 

The implementation of the control system was divided into two parts. To achieve recognition of performed movement a classifier was trained, however, this only produces a recognition of a movement and does not reflect the intensity of which the movement is being performed with. Therefore, following the recognition of performed movement a linear regression model was implemented to achieve proportional control. 

\subsubsection{Classification of Movement}

Real-time classification of movements was accomplished by implementing a LDA classifier. As presented in \secref{patter} the classifier needed to be trained using data from each movement. Hence, the five features extracted for each of the 40 $\percent$, 50 $\percent$ and 70 $\percent$ fraction of the pMVC for one movement were assembled into one labeled training matrix. The same was done for the three remaining movements. A fifth class was labeled rest and its training matrix only contained the features from the single rest acquisition.  

3 intensities for each movement 
4 movements + 1 rest 
5 features 
8 channels 

\subsubsection{Proportional control}  


\subsection{Assessment of Subject Control}

After the acquisition of data training data and the training of the classifier, two stages were the subject could familiarize themselves with the control and test how well they were able to use the control system, were implemented. It was highly critical that the subject was able to achieve sufficient control such that it would not be due to insufficient control, that a subject was not able to reach a target during test of the feedback schemes. However, as the classifier only had five classes, representing five very independent movements, to distinguish between, the classifier achieved high classification accuracy. Therefore, the need for subject training could be kept to a minimum. \fxnote{This was furthermore supported by the results from pilot studies}.  

\subsubsection{Familiarization with Control}

At first, the subject was presented with an image of the grid, which can be seen in \figref{make ref}, in which it would be possible to move a cursor around inside in order to navigate. The cursor represents different 



\subsubsection{Target Reaching Test}
