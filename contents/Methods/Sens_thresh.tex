
\section{Determination of Sensory Thresholds}

Having completed the initial step of the experiment, where the control system was build and the subject's ability to use the control was assessed, the next step was to determine the subject's sensory thresholds, which would be used to convey feedback information. \\
In order to provide meaningful sensory feedback to the subject, a range of distinguishable sensory thresholds had to be determined for the subject. As presented in \secref{sec:vp} the virtual prosthesis has a range of one to four states. Hence, four thresholds based on amplitude values were determined to accommodate four levels of feedback in the amplitude scheme. Furthermore, since the sensory sensitivity varies across different locations of the circumference of the arm, the sensory thresholds had to be determined for each individual pad in the electrode armband. Sensory thresholds were found by slowly increasing the amplitude, while fixating the pulse width and frequency at 500 $\mu s $ and 50 Hz, respectively. \\
In the first round, the lowest threshold was determined, which from now on will be referred to as the perception threshold. For each pad, the amplitude was set to start at zero and then increase by 100 $\mu A $ once per second. The subject was instructed in reporting when electrical stimulation could be sensed and that the subject was sure the activated pad was the origin of the perceived stimulation. The pad was deactivated and reactivated once more with the reported sensory threshold amplitude for a second verification. This process was carried out for each pad starting from pad 1 to 16. Subsequently, the subject was presented the set amplitudes in each pad, such that the sensation in the current pad was compared to the prior. This was carried out such that determined amplitudes could be readjusted enabling more homogeneous sensations across all pads.  \\
In the second round, the fourth level thresholds, referred to as tolerance threshold, were determined using the same procedure as in round one. However, using the starting amplitude value set in round one. The amplitude was set to increase by 200 $\mu A $ once per second. The amplitude was set to increase until the subject reported that the sensation was becoming unpleasant, the stimulation was causing functional muscle activation or a maximum of 10000 $\mu A $ was reached. Again the intensities were readjusted for homogeneous sensitivity. \\ 
Using the determined perception and tolerance thresholds, the second and third level thresholds were calculated for each pad $i$, such that

\begin{equation}
2~lvl(i) = perception(i) + (\frac{1}{3} * (tolerance(i) - perception(i)))
\end{equation}

\begin{equation}
3~lvl(i) = tolerance(i) - (\frac{1}{3} * (tolerance(i) - perception(i)))
\end{equation}


       