\chapter{Introduction}

The loss of an upper limb can be incredibly traumatic and life changing event with the consequence of a significantly reduced quality of life due to restrictions in function, sensation and appearance \cite{Schofield2014,Ostlie2011}. The loss is additionally linked to the development of multiple mental health disorders \cite{Ostlie2011}.
In an effort to restore pre-trauma functionality, prosthetics of various functionality and complexity have been introduced to replace the missing limb \cite{Geethanjali2016}. However, despite advancements in prosthetic technologies every 1/4 choose to abandon their myoelectric prosthetic device \cite{Biddiss2007a}. An explanation for the low user satisfaction should be found in the lack of exteroceptive and proprioceptive feedback provided by commercially available devices \cite{Peerdeman2011}. Presently, merely one device (VINCENT evolution 2, Vincent Systems Gmbh, DE), is commercially available providing the user with feedback information of grasping force, through a feedback interface \cite{Systems2005}. \\    
%
The missing sensory feedback can cause the prosthetic hand to feel more unnatural and awkward \cite{Pamungkas2015}, thus the user solely have visual feedback to rely on \cite{Stephens-Fripp2018,Pamungkas2015}, a need prosthetic users have shown a strong desire to decrease \cite{Atkins1996}. In a survey by Peerdeman et al. \cite{Peerdeman2011} it was found that secondly to providing proportional grasp force feedback, providing positional feedback was of highest priority. Visual independence can be achieved by providing the user with proprioceptive information through somatosensory feedback, possibly facilitating the prosthetic device to be adopted by the user as an integrated part of their body, enhancing the feeling of embodiment and restoring the once physiologically closed loop \cite{Stephens-Fripp2018,Xu2016,Strbac2016,Geng2012}. \\
%
Various means of recreating the sensory feedback has been sought through either invasive and non-invasive approaches, translating information from sensors in the prosthesis to new sensory sites. Invasive methods, termed somatotopically feedback aims to recreate the localization of the prior sensory experience by directly stimulating specific nerves in the residual limb \cite{Schofield2014,Stephens-Fripp2018}.
Substitution feedback utilize various tactors (pressure, vibrational, temperature, electrotactile, etc.) and their use can either be modality matched using pressure as a substitute for grasp force \cite{Godfrey2017} or non modality matched via vibration for grasp force \cite{Ninu2014,Nabeel2016}. 
Electrotactile feedback uses small electrical currents to activate skin afferents eliciting sensory sensations which can be modulated in multiple parameters such as pulse width, amplitude and frequency to convey feedback information along with providing the possibility of using multiple feedback channels \cite{Geng2012}. The relevance of employing multi-channel feedback is justified by commercially available upper limb prosthetics have multiple degrees of freedom (DoF's) \cite{Cordella2016}. \\
%
The use of electrotactile feedback has earlier proven useful in cases of restoring force feedback through pressure sensors on a prosthetic hand or by the touch on artificial skin \cite{Hartmann2014,Franceschi2015}. However, the possibilities of electrotactile feedback have also been investigated in the case of improving prosthetic control. Strbac et al. \cite{Strbac2016} presented a novel electrotactile feedback stimulation interface, which could be used to convey information about the current state of a multi-DoF prosthesis. The system comprised of four different dynamic stimulation patters communicating the states of four different DoF's through a 16 multi-pad array electrode. The state of three different DoF's were communicated by altering the electrodes activated in a specific pattern and the fourth, grasp force, by modulating the stimulation frequency. Tests of the stimulation design showed that six amputees were able to recognize the stimulation pattern of the four DoF's with an average accuracy of 86 \percent. \cite{Strbac2016} \\   
%
(Some great build up) investigate which types of electrotactile feedback is perceived more intuitive when conveying proprioceptive sensory feedback of the current prosthetic state. In this study we will present two different stimulation protocols to convey the before mentioned information; one based on activation of differently spatially located electrode pads, and another based on delivering different levels of amplitude.      


 
