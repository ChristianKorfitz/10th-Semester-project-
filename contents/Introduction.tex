\chapter{Introduction}

The loss of an upper limb can be incredibly traumatic and life changing event with the consequence of a significantly reduced quality of life due to restrictions in function, sensation and appearance \cite{Schofield2014,Ostlie2011}. The loss is additionally linked to multiple mental health disorders \cite{Ostlie2011}.
In an effort to restore pre-trauma functionality, prosthetics of various functionality and complexity have been introduced to replace the missing limb \cite{Geethanjali2016}. However, despite advancements in prosthetic technologies only 50 $\percent$ to 60 $\percent$ of hand amputees wear a prosthetic device \cite{Stephens-Fripp2018}. An explanation for the low user satisfaction should be found in the lack of exteroceptive and proprioceptive feedback provided by commercially available devices \cite{Peerdeman2011}. Presently, merely one device (VINCENT evolution 2, Vincent Systems Gmbh, DE), is commercially available providing the user with feedback information of grasping force, through a feedback interface \cite{Systems2005}. \\    
The missing sensory feedback can cause the prosthetic hand to feel more unnatural and awkward \cite{Pamungkas2015}, thus the user solely have visual feedback to rely on \cite{Stephens-Fripp2018,Pamungkas2015}, which prosthetic user have shown a strong desire to decrease \cite{Atkins1996}. In a survey by Peerdeman et al. \cite{Peerdeman2011} it was found that secondly to providing proportional grasp force feedback, providing positional feedback was of highest priority. Visual independence can be achieved by providing the user with proprioceptive information through somatosensory feedback, possibly facilitating the prosthetic device to be adopted by the user as an integrated part of their body, enhancing the feeling of embodiment and restoring the once physiologically closed loop \cite{Stephens-Fripp2018,Xu2016,Strbac2016,Geng2012}. \\
Various means of recreating the sensory feedback has been sought through either invasive and non-invasive approaches translating information from sensors in the prosthesis to new sensory sites. Invasive methods, termed somatotopically feedback aims to recreate the prior sensory experience by directly stimulating specific nerves in the residual limb \cite{Schofield2014,Stephens-Fripp2018}.
Substitutionary feedback can either be modality matched using pressure as a substitute for grasp force \cite{Godfrey2017} or non modality matched via vibration for grasp force \cite{Ninu2014,Nabeel2016}.
As electrotactile feedback offers modulating multiple parameters such as pulse width, amplitude and frequency to convey feedback information along with the possibility of using multiple feedback channels it seems ideal to utilize these perks for proprioceptive sensory feedback of the multi degree of freedom prosthetic state.  

electrotactile feedback schemes

Based on the current work it would reasonable to investigate which types of electrotactile feedback is perceived more intuitive when conveying proprioceptive sensory feedback of the current prosthetic state. In this study we will present two different stimulation to convey the before mentioned information; one based on activation of differently spatially located electrode pads, and another based on delivering different levels of amplitude.      


 
