\chapter{Project Objective}

In summary, it would be beneficial for the understanding of chronic pain to gain further knowledge on the brain mechanisms associated with acute pain. As different individuals report a wide variety in pain sensitivity when given identical pain stimuli, an examination on if the subjective reports correlate with intensity in brain activation in brain regions associated with pain would be of high interest. When using 3.0 tesla MRI scanners to indirectly measure brain activity through BOLD fMRI sequences, more of the activity will be captured compared to using 1.5 tesla scanners. The images will on the other hand be more prone to contain noise. For this reason proper preprocessing is needed to separate noise from signal of interest before analyzing the images. During the recent years more preprocessing methods have been developed in the field of fMRI as alternatives to the standard method. One of these is FIX, which with the use of MELODIC has shown great noise removal capabilities in resting state fMRI. A comparison of FIX and the standard method, presented in \secref{art}, on a large set of participants that underwent a noxious stimuli task would be of great interest in the field of pain research, as this could provide insight on the advantages/disadvantages of this preprocessing method. This introduces the study objective:

\begin{center}

\textit{Evaluate the performance of using the standard preprocessing pipeline compared to incorporating FIX in the pipeline for noise removal in images acquired with a BOLD fMRI sequence in the context of individual cerebral activation as response to noxious stimuli.}

\end{center}

From the project objective the following primary and secondary study questions are sought answered: 

\begin{itemize}
	\item Primary: Does incorporating FIX in the fMRI preprocessing pipeline generate a higher signal-to-noise ratio?
	
	\item Secondary: Is the detection of individual differences improved when using FIX preprocessed data compared to standard preprocessed data?
	   
	 
\end{itemize}

