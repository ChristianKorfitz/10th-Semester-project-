\section{Prosthetic control system}
The extracted features were used to train a sequential proportional control system. For sequential control, a LDA classifier was used and for proportional control multiple linear regression models were used. The following section will address the fitting of the control system. 

\subsection{Classification model}
A feature set was calculated for each of the eight electrode pairs and subsequently concatenated resulting in a 48-dimensional feature matrix that was provided to the classifier. It was chosen to implement a LDA classifier due to it being quick to train, while still yielding robust control \cite{Englehart2003}. The LDA classifier determined decision boundaries by maximizing the distance between centroids of the movement class feature values. Such decision boundaries were defined as a linear combination of feature values, where the output was posterior probabilities for each movement class. The decision rule was that the movement class with the highest probability would decide the determined motor function. The classifier was trained in distinguishing between five classes: wrist supination, wrist pronation, closed hand, opened hand and rest.  

\subsection{Proportional control}
The proportional control model provided the control system with an actuation velocity proportional to the contraction intensity in a direction based on which movement class that was decided on by the classifier. This was achieved by training four multiple linear regression models: one for each active movement class. The mean absolute value (MAV) was calculated in windows for all electrode pairs and provided to the regression model as independent values, where the MAV scaled relative to the pMVC was provided as dependent values. During online control, the output was limited to a maximum value of 1; corresponding to the intensity of the pMVC. The distance of 1 corresponded to 1 cm on the computer screen, thus, the maximum speed of the virtual prosthesis was 1 cm per update (100 ms). A full DoF would be completed from one extremity to another in two seconds, thus, achieving an actuation velocity similar to the commercially available Bebionic prosthesis \cite{Belter2013}. A second restriction implemented was that a movement had to be performed with >15 $\percent$ contraction intensity, for the virtual prosthesis to be actuated. This was included to get a more stable performance at rest. 