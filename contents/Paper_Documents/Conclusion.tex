This study investigated the intuitiveness of two novel electrotactile feedback configurations communicating proprioceptive information of a two DoF closed-loop myoelectric prosthesis: one modulating the spatial activation of electrode pads and one modulating the current amplitude. The evaluation tests showed that even with minimal training (< 30 minutes) a mean success rate of 93 \% $\pm$ 6 \% and 87 \% $\pm$ 11 \% can be achieved for the amplitude and spatial modulated configurations, respectively; and along with subjects reporting that both feedback schemes were easily comprehensible, the developed feedback schemes can be deemed highly intuitive. As the stimulation setup demanded scarce space, it could be easily integrated in a two DoF myoelectric prosthesis, potentially enhancing the prosthesis embodiment in users. Moreover, especially the amplitude feedback scheme had the potential to convey the position states even less discretely, which would further increase the naturalness of use. 