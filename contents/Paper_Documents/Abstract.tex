The implementation of intuitive and meaningful proprioceptive feedback of position states in myoelectric prosthesis is an important aspect in enhancing embodiment and user satisfaction, hence lowering the demand for visual attention for prosthetic control in everyday tasks. Therefore, two different configurations for conveying position state information of wrist rotation and hand aperture through electrotactile stimulation were developed and evaluated in a simulated closed-loop prosthesis. A spatially-based configuration was made conveying information by changing the activation of pads in an electrode array placed circumferentially around the non-dominant arm. The other scheme was amplitude-based and used various levels of amplitude from specific electrode pads to convey information of the position state of the prosthesis. 14 able-bodied subjects were evaluated through a Fitts' Law inspired target reaching test following a minimal training session.
The amplitude-based and spatially-based configurations yielded mean completion rates of 93 \% $\boldsymbol{\pm}$ 6 \% and 87 \% $\boldsymbol{\pm}$ 11 \%, respectively. The amplitude feedback configuration yielded a slightly higher completion rate (p = 0.044) than the spatially-based and was also preferred by 64 \% of the subjects. However, with such high completion rates both schemes can be regarded intuitive and was subjectively reported to be useful and easily comprehensible. This manifests that both developed feedback configurations allow subjects to perceive two feedback variables at the same time, despite being implemented in a compact stimulation interface. 
 
 
 
 %The completion rate for visual feedback (99 $\% \pm 2\%$) was