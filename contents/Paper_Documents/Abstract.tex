The restoration of intuitive and meaningful proprioceptive feedback of myoelectric prosthetic state is an important task to enhance embodiment and user satisfaction, hence lowering the demand for visual attention for prosthetic control in everyday tasks. Therefore, two different configurations for conveying two DoF prosthetic state information of wrist rotation and hand opening through electrotactile feedback stimulation were developed and evaluated in a simulated closed-loop prosthesis. A spatially based configuration was made conveying information by changes the activation of pads in an electrode array placed circumferentially around the contra-lateral arm. The other, amplitude based, used various levels of amplitude to specific pads to convey information of the prosthetic state. 

14 able-bodied subjects were trained and evaluated through a blinded target reaching test in using both feedback configurations following a minimal training session. The completion rate for visual feedback (99 $\%$) significantly outperformed both the spatially based (87 $\%$) and the amplitude based (93 $\%$) configurations. The amplitude feedback configuration yielded a significantly higher completion rate (p = 0.044) than the spatially based and was also preferred by 64 $\%$ of the subjects. However, both feedback schemes were reported to be useful and intuitive manifesting that both configurations are ready to be tested in less discriminative cases.    
