The restoration of intuitive and meaningful proprioceptive feedback of myoelectric prosthetic state is an important aspect in enhancing embodiment and user satisfaction, hence lowering the demand for visual attention for prosthetic control in everyday tasks. Therefore, two different configurations for conveying two DoF prosthetic state information of wrist rotation and hand opening through electrotactile feedback stimulation were developed and evaluated in a simulated closed-loop prosthesis. A spatially based configuration was made conveying information by changes the activation of pads in an electrode array placed circumferentially around the contra-lateral arm. The other scheme was amplitude based and used various levels of amplitude from specific electrode pads to convey information of the prosthetic state. 

14 able-bodied subjects were evaluated through a Fitts' Law inspired target reaching test following a minimal training session. The completion rate for visual feedback (99 $\%$) was significantly higher than both the spatially based (87 $\%$) and the amplitude based (93 $\%$) configurations. The amplitude feedback configuration yielded a significantly higher completion rate (p = 0.044) than the spatially based and was also preferred by 64 $\%$ of the subjects. However, with such high completion rates both schemes can be regarded intuitive and was subjectively reported to be useful and easily comprehensible, which manifests that both configurations are ready to be expanded to less discrete use.    
