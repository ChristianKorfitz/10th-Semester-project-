
\subsection{Determination of Electrotactile Sensory Thresholds}


Providing meaningful sensory feedback required determination of four distinguishable subject specific electrotactile thresholds. Threshold were made solely amplitude dependent by keeping pulse width and frequency constant at 500 $\mu s$ and 50 $Hz$, respectively. 1st level thresholds, termed perception thresholds, were determined for each pad by starting the amplitude at 0 $\mu A$ and then increasing in steps of 100 $\mu A$ per second. The subject was instructed to report when stimulation could be perceived confidently. Subsequently, the intensities were readjusted by comparing the sensation in each pad to the neighboring to achieve homogeneous sensations across all pads. \\
4th level thresholds, termed tolerance threshold, were set using the same approach, but initiating the amplitude value at the 1st level thresholds and increasing the amplitude in steps of 200 $\mu A$ per second. The thresholds were determined when the subject reported that the sensation was on the onset to getting unpleasant, the stimulation was becoming functional or a maximum of 10,000 $\mu A$ was reached. Intensities were again readjusted to achieve homogeneous sensations. Throughout the process of determining threshold the subject was was faced away from the screen. Intermediate threshold levels 2 and 3 were calculated for the $i^{th}$ pad based the perception and tolerance threshold as following. 
\vspace{-0.2cm}
{\small 
\begin{equation}
2~lvl(i) = perception(i) + (\frac{1}{3} \cdot (tolerance(i) - perception(i)))
\end{equation}}
{\small
\begin{equation}
3~lvl(i) = tolerance(i) - (\frac{1}{3} \cdot (tolerance(i) - perception(i)))
\end{equation}
}