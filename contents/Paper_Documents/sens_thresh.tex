
\subsection{Determination of Electrotactile Sensory Thresholds}


Providing meaningful sensory feedback required determination of four distinguishable subject specific electrotactile thresholds. A multichannel electrotactile stimulation feedback device (MaxSens, Tecnalia, Spain) generating biphasic pulses was connected to a standard desktop PC for individual control of pad activation. Thresholds were made solely amplitude dependent by keeping pulse width and frequency constant at 500 $\mu s$ and 50 $Hz$, respectively.  1st level thresholds, termed perception thresholds, were determined for each pad by initializing the amplitude at 0 $\mu A$ and then increase it in steps of 100 $\mu A$ per second. The subject was instructed to report when stimulation could be perceived confidently. Subsequently, the intensities were readjusted by comparing the sensation intensity in neighboring pads to achieve homogeneous sensations across all pads. 

4th level thresholds termed tolerance threshold, were set using the same approach, however, the amplitude level was initialized at the perception threshold and increased in steps in steps of 200 $\mu A$ per second. The thresholds were determined when the subject reported that the sensation was on the onset of getting unpleasant, the stimulation was becoming functional or a maximum of 10,000 $\mu A$ was reached. Intensities were again readjusted to achieve homogeneous sensations. Throughout the process of determining thresholds, the subject was faced away from the screen to avoid bias from observing the visual increase of amplitude levels. Intermediate threshold levels 2 and 3 were calculated for the $i^{th}$ pad based on the perception and tolerance threshold as: 

	\begin{equation}
	2~lvl(i) = percep.(i) + (\frac{1}{3} \cdot (tol.(i) - percep.(i)))
	\end{equation}

	\begin{equation}
	3~lvl(i) = tol.(i) - (\frac{1}{3} \cdot (tol.(i) - percep.(i)))
	\end{equation}
