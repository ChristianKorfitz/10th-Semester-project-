\subsection{Subject Control Training and Evaluation} \label{sec:pa:subjectcontrol}
The subjects were initially trained in controlling the virtual prosthesis via visual feedback. It was crucial for subjects to achieve robust control for the feedback configurations to be able to be evaluated in a closed-loop prosthetic control system. The subjects' control abilities were assessed empirically during trainings and quantitatively through a Fitts' Law inspired target reaching test. If a subject did not have a completion rate above 90 $\%$ and a mean time to reach a target at below 10 seconds the subject would be excluded. 

The subject control training was divided into two runs of three minutes with a different prosthesis representation in each training. In the first training, the prosthesis was represented as a black cursor as seen in figure \ref{fig:pa:gridmap}. The cursor position would update continuously with each control system output. In the second training, the cursor was invisible, and the prosthesis representation was instead the square containing the cursor being highlighted. This discretized prosthesis representation was implemented to equalize the visual and sensory feedback. This prosthesis representation was used in the remaining training/test runs with visual feedback. During both training phases the subjects were instructed in practicing the ability to move the cursor in a desired direction and to transition from movement to rest. 

During the target reaching test, the subjects had to reach targets (highlighted grid squares) visualized in a randomized order. The subjects had to match the discretized virtual prosthesis with the target and dwell in that position for 1.5 seconds for it to be deemed reached. The subjects had 30 seconds to reach a target. If either a target was reached or the time limit was reached, the virtual prosthesis would reset in neutral position. The test was finished when all grid squares had been highlighted, making a total of 24 targets. 