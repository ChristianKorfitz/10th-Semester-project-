

The loss of an upper limb can be an incredibly traumatic and life-changing event with the consequence of a significantly reduced quality of life due to restrictions in function, sensation and appearance \cite{Schofield2014,Ostlie2011}. 
%The loss is additionally linked to the development of multiple mental health disorders \cite{Ostlie2011}.
In an effort to restore pre-trauma functionality, prosthetics of various functionality and complexity have been introduced to replace the missing limb \cite{Geethanjali2016}. However, despite advancements in prosthetic technologies 25\% of users choose to abandon their myoelectric prosthetic device \cite{Biddiss2007a}. A major reason for the low user satisfaction is found in the lack of exteroceptive and proprioceptive feedback provided by commercially available devices \cite{Schofield2014,Peerdeman2011}. Presently, merely one commercially available device (VINCENT evolution 2, Vincent Systems Gmbh, DE), provides the user with feedback information of grasping force through a feedback interface \cite{Systems2005}. 
    
%
The missing sensory feedback can cause the prosthetic hand to feel more unnatural and awkward \cite{Pamungkas2015}. Furthermore, the user mainly relies on visual feedback \cite{Pamungkas2015,Stephens-Fripp2018}, which is a need prosthetic users have shown a strong desire to decrease in order to enhance easiness and naturalness of use \cite{Atkins1996}.
In a survey by Peerdeman et al. \cite{Peerdeman2011}, it was found that secondly to receiving proportional grasp force feedback, prosthetic positional state feedback was of the highest priority. Visual independence can be achieved by providing the user with proprioceptive information through somatosensory feedback. This might facilitate the prosthetic device to be adopted by the user as an integrated part of their body, enhancing the feeling of embodiment and restoring the once physiologically closed loop \cite{Stephens-Fripp2018,Xu2016,Strbac2016,Geng2012}. 

%
Various means of recreating the sensory feedback has been sought through either invasive and non-invasive approaches that translate information from sensors in the prosthesis to new sensory sites. Invasive methods, termed somatotopical feedback, aim to recreate the localization of the prior sensory experience by directly stimulating the nerves, which conveyed that particular sensory modality in the lost limb. This is, however, a complicated solution and multiple aspects, like long term effect, have yet to be investigated. \cite{Schofield2014,Stephens-Fripp2018}. 
Substitution feedback utilizes various tactors (pressure, vibrational, temperature, electrotactile, etc.) and their use can either be modality matched using e.g. pressure as a substitute for grasp force \cite{Godfrey2017} or non-modality matched via e.g. vibration for grasp force \cite{Ninu2014,Nabeel2016}. 
Electrotactile feedback uses small electrical currents to activate skin afferents eliciting sensory sensations, which can be modulated in multiple parameters such as pulse width, amplitude, and frequency to convey feedback information along with the possibility of using multiple feedback channels \cite{Geng2012}. As commercially available upper-limb prosthetics have multiple degrees of freedom (DoF's) \cite{Cordella2016} the need for multiple feedback channels is present to accommodate the amount of information which needs to be provided in a meaningful way. 

%
In cases where two information variables are being conveyed e.g. grasping force and hand aperture using frequency and amplitude modulation in electrotactile stimulation \cite{Prior1976} or pulse interval and stimulation frequency in vibrotactile stimulation \cite{Chatterjee2008}, results have shown that one stimulator is not sufficient for users to distinguish between two modalities. In 2014, Witteveen et al. \cite{Witteveen2014} provided sensory feedback of grasping force and hand aperture through a single vibrator and an array of vibrotactile actuators, respectively. Results showed that identification of stiffness for four virtual objects was around 60 $\%$. Although the percentage was rather low, the feedback configuration proved better compared to no feedback showing that multichannel feedback helps distinguishability when conveying feedback of more than one information variable. \cite{Witteveen2014} However, the use of multiple vibrotactile actuators might be less feasible and practical to implement in prosthetics, due to their size and greater power consumption compared to electrotactile stimulation.  

%
The flexibility of electrotactile stimulation makes is desirable and its use has earlier been proven useful in cases of conveying force feedback from pressure sensors on a prosthetic hand or from sensors in artificial skin \cite{Hartmann2014,Franceschi2015}. However, the possibilities in electrotactile feedback have also been investigated with regards to communication information on states of a multi DoF prosthesis. Strbac et al. \cite{Strbac2016} presented a novel electrotactile feedback stimulation interface, which could be used to convey information about the current state of a multi-DoF prosthesis. The system was comprised of four different dynamic stimulation patterns communicating the states of four different DoF's through a 16 multi-pad array electrode. The state of three different DoF's were communicated by altering the electrodes activated in a specific pattern. The fourth pattern communicated grasp force by modulating the stimulation frequency. Tests of the stimulation design showed that six amputees were able to recognize the stimulation pattern of the four DoF's with an average accuracy of 86 \%. \cite{Strbac2016} However, it was not tested how well these stimulation patterns was aiding the user when combined with prosthetic control.  \\   
%
To the authors' knowledge, no one has fully closed the neural afferent/efferent loop, when investigating the usability of electrotactile feedback for restoring proprioceptive aspects during the use of a myoelectric prosthesis. Furthermore, based on the multiple parameters that can be modulated in electrotactile feedback, the question of which parameters that are most useful to convey tactile information on prosthetic motion states, is still unanswered. This study will, therefore, investigate how different electrotactile feedback modalities support prosthetic control when conveying proprioceptive sensory feedback of prosthetic states using a stimulation setup. Two novel stimulation configurations that delivered feedback regarding motion states of a two DoF virtual myoelectric prosthesis were investigated; one based on spatial activation of differently located pads in an electrode array, and one based on modulating the current amplitude of the electrode pads.

In Section II the study design will be presented, followed by an introduction of the two novel feedback configurations derived. Subsequently, the closed-loop prosthesis, control system and experimental protocol will presented. Results of the experiment will be reported in section III. Finally, the significance of this study and its results will be presented in Sections IV and V. 
















