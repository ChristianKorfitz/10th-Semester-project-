
\subsection{Sensory Feedback Training}

Following the subject training of prosthetic control, the subject was trained in understanding a sensory feedback scheme. The sensory feedback training was divided into two phases: familiarization and reinforced learning. 

The familiarization phase provided the subjects with a short and controlled introduction to the scheme. The cursor was visualized and moved by the investigator from the neutral position to a designated state incorporating the transition from one square to the next, thus, presenting the subject with the coherence between feedback variable level and position state for a designated state. Feedback levels in the top and bottom row and the middle column were presented actively, while the remaining 12 levels were presented indirectly as transition levels. Moving to feedback levels of 4th level hand aperture combined with either 1st or 2nd level wrist pronation and supination was done by first moving in the rotational DoF. Time spend in designated states was approximately four seconds and time spend in transition states was approximately two seconds. Recognition of single DoF position states was assessed to be most crucial for comprehension, hence, these were favored in the familiarization phase. 

For the reinforced learning phase the subject was asked to face away from the screen. The cursor was directed to a designated state and the subject then had to report what the current position state was based solely on the felt feedback variable. If the subject answered correctly, the cursor was reset to the neutral position and the cursor was moved to a new target. If the subject answered incorrectly, the correct state would be communicated to the subject before continuing. Each position state would be presented once and be moved to by taking the optimal path (move the cursor fully in one DoF before the other). However, which DoF the cursor would move in first was varied. Hence, the subject could utilize the transitions made when guessing the current state. The order of the designated states was predetermined by the investigators. Time spend in transition states was approximately two seconds. When all 24 position states had been trained, the subject was given a short break before repeating the reinforced learning. However, the order and paths were changed for the second run.