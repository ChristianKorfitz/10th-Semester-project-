\subsection{Myoelectric Prosthetic Control}

In order for a subject to be able to control the virtual prosthesis online, the prosthetic control system needed to be trained with acquired EMG signals. For EMG data acquisition the Myo Armband (MYB) from Thalmic labs was used, which contained eight dry stainless steel electrode channels embedded on the inside of the armband. Furthermore, it could communicate wirelessly to external devices via a Bluetooth 4.0 unit, making it a highly practical recording device with minimum preparation time needed. However, it had a fixed sample rate of 200 Hz with the exclusive analogue filter being a 50 Hz notch filter. The EMG signal was, therefore, aliased. A study by Mendez et al. \cite{Mendez2017} showed, however, a similar mean classification accuracy of nine hand gestures in a Linear Discriminant Analysis (LDA)-based classifier, when comparing data acquired with electrodes that covered the entire EMG spectrum and the MYB acquired data. This justified the use of the MYB and only a 10 Hz cut-off second order Butterworth high-pass filter was implemented digitally to remove low frequency artefacts.

To account for the delay until steady state motions were reached, it was desired to train the prosthetic control system with both transient and steady state EMG data from each movement \cite{Boschmann2013}. To achieve this, the subjects were to follow a trapezoidal trajectory, where they controlled a cursor that moved horizontally with time and vertically with EMG intensity. The recording was 11 seconds, where the trajectory had an incline/decline of three seconds and a plateau of five seconds, representing the transient and steady state, respectively. However, only data from the last second of the incline and first seconds of the decline was used to train the classifier to avoid active motion classes being misclassified with rest. The trajectory and cursor position was scaled relative to an initially recorded prolonged maximum voluntary contraction (pMVC) of 15 seconds. When performing the pMVC the subject was instructed in eliciting a strong voluntary contraction that could be held steady for 15 seconds. Data was acquired from three recordings per movement, where the plateau was 40 $\%$, 50 $\%$ and 70 $\%$ of the pMVC's, respectively. A last recording of 15 seconds rest was also performed.

For feature extraction, space-domain features designed by Donovan et al. \cite{Donovan2017} were applied. These were developed to enhance the classification accuracy when using the MYB for data acquisition by exploiting the relationship between EMG signals from neighboring electrode channels. The four non-redundant space-domain features of Scaled Mean Absolute Value, Correlation Coefficient, Mean Absolute Difference Normalized, Scaled Mean Absolute Difference Raw were extracted, along with the Hudgins feature Waveform Length to obtain an indirect representation of the frequency content \cite{Hudgins1993}. Both offline and online features were extracted in windows of 200 ms with a 50 $\%$ overlap to obtain quick update time, while preserving robust classification accuracy \cite{Menon2017}.

The extracted features were used to train a sequential proportional control system. For sequential control, a LDA classifier was used and for proportional control multiple linear regression models were used.

The classifier was trained in distinguishing between five classes: wrist supination, wrist pronation, closed hand, opened hand and rest. A feature set was calculated for each of the eight electrode channels and subsequently concatenated resulting in a 40-dimensional feature matrix that was provided to the classifier. It was chosen to implement a LDA classifier due to it being quick to train, while still yielding robust control \cite{Englehart2003}.  

 The proportional control model provided the control system with an actuation velocity proportional to the contraction intensity in a direction based on which movement class that was decided on by the LDA classifier. This was achieved by training four multiple linear regression models: one for each active movement class. The mean absolute value (MAV) was calculated in windows for all electrode channels and provided to the regression model as independent values, where the MAV scaled relative to the pMVC was provided as dependent values. During online control, the output was limited to a maximum velocity of 1 cm on the computer screen; corresponding to the intensity of the pMVC. Thus, the maximum speed of the virtual prosthesis was 1 cm per update (100 ms). A full DoF would be completed from one extremity to another in two seconds, thus, achieving an actuation velocity similar to the commercially available Bebionic prosthesis \cite{Belter2013}. A second restriction implemented was that a movement had to be performed with >15 $\%$ contraction intensity, for the virtual prosthesis to be actuated. This was included to get a more stable performance at rest. 