\subsection{Data Acquisition}

In order for a subject to be able to control the virtual prosthesis online, the prosthetic control system needed to be trained with previously acquired EMG signals. For EMG data acquisition the Myo Armband (MYB) from Thalmic labs was used, which contained eight dry stainless steel electrode pairs embedded on the inside of the armband. Furthermore, it could communicate wirelessly to external devices via a Bluetooth 4.0 unit, making it a highly practical recording device with minimum preparation time needed. However, it had a fixed sample rate of 200 Hz with the exclusive analogue filter being a 50 Hz notch filter, thus, making the EMG signals prone to aliasing. A study by Mendez et al. \cite{Mendez2017} showed, however, a similar mean classification accuracy of nine hand gestures in a LDA classifier, when comparing data acquired with electrodes that covered the entire EMG spectrum and the MYB acquired data. This justified the use of the MYB and only a 10 Hz cut-off second order Butterworth high-pass filter was implemented digitally to remove low frequency artefacts.

To account for the delay until steady state motions were reached, it was desired to train the prosthetic control system with both transient and steady state EMG data from each movement \cite{Boschmann2013}. To achieve this, the subjects were to follow a trapezoidal trajectory, where they controlled a cursor that moved horizontally with time and vertically with EMG intensity. The recording was 11 seconds, where the trajectory had an incline/decline of three seconds and a plateau of five seconds, representing the transient and steady state, respectively. However, only data from the last second of the incline and first seconds of the decline was used to train the classifier to avoid active motion classes being misclassified with rest. The trajectory and cursor position was scaled relative to an initially recorded prolonged maximum voluntary contraction (pMVC) of 15 seconds, which was set to 1. Data was acquired from three recordings per movement, where the plateau was 40 $\%$, 50 $\%$ and 70 $\%$ of the pMVC's, respectively. A last recording of 15 seconds rest was also performed.

\subsection{Feature Extraction}
Features were extracted from the acquired EMG signals to expand the amount of information used to train the classifier in the prosthetic control system. Due to the risk of the EMG signals being aliased, features representing frequency content might lose fidelity. In a study by Donovan et al. \cite{Donovan2017}, the classification accuracy of space-domain features exploiting the relationship between EMG signals from neighboring electrode pairs in the MYB were compared with the commonly used Hudgins features in a LDA classifier. The use of space-domain features yielded a 5 $\%$ higher accuracy than the Hudgins features with EMG data acquired from the MYB. It was therefore chosen to use the four non-redundant features of Scaled Mean Absolute Value, Correlation Coefficient, Mean Absolute Difference Normalized, Scaled Mean Absolute Difference Raw derived by Donovan et al., along with the Hudgins feature Waveform Length to represent frequency content indirectly \cite{Hudgins1993}, resulting in a total of five features. Both offline and online features were extracted in windows of 200 ms with a 50 $\%$ overlap to obtain quick update time, while preserving robust classification accuracy \cite{Menon2017}. 
