Two intuitive electrotactile feedback schemes were developed for a two DoF velocity-based virtual prosthesis: one spatially modulated and one amplitude modulated. The schemes were integrated in an easy implementable 16 pad electrode array and tested in combination with sequential proportional myoelectric control. Unique sensory feedback was provided for four levels of prosthetic states in single DoF's and for 16 prosthetic states representing combined DoF's. The objective was to investigate the usability of the developed feedback schemes when removing visual dependency.

From the metrics extracted describing the subjects' performance in the evaluation test, only completion rate indicated a slight  dominance in favor of the amplitude scheme compared to the spatial scheme (p-value = 0.044). However, with a mean completion rate of 93 \% $\pm$ 6 \% and 87 \% $\pm$ 11 \%, respectively, both feedback schemes can be deemed intuitive to utilize in combination with myoelectric control when removing visual dependency. Considering that these completion rates were obtained from a minimal training protocol (training time per scheme < 30 minutes), a completion rate similar to visual feedback (99 $\% \pm 2\%$) might be achieved if more training blocks were included.
Compared to the results of Strbac et al. \cite{Strbac2016} the results of recognizing more complex four DoF stimulation patterns which achieved a success rate of 99 $\% \pm 3\%$ for able-bodied subjects, the usability of the derived schemes seems lower. However, Strbac et al. did not test the usability of their feedback schemes in combination with control. Furthermore, recognizability was only investigated for each DoF independently and not in combinations as in this study. We speculate, that eliminating these variables, similar results would be achieved when subjects were given adequate training time.       

In the reinforced learning, the mean success rate for the spatial scheme was 73 \%  $\pm$ 17 \% and 78 \%  $\pm$ 16 \% for the amplitude scheme. This was a notably lower success rate than obtained from the closed-loop evaluation tests. This could indicate that when put into the intended application, a higher understanding of the feedback can be accomplished. If the training block had the same duration, but was solely closed-loop-based, an even higher success rate might have been achieved in the evaluation tests. 

\subsection{Sensory Thresholds}
Some subjects reported that it was difficult to separate levels in both DoF's in the spatial scheme, due to a notable difference in sensation intensity between levels. A different approach in the determination of sensory threshold levels might have removed this confusion in the spatial feedback. The amplitude levels were determined by setting the threshold level for the electrode pads in a consecutive order. This might have caused a slight adaptation in the sensory perception of the subjects, which distorted the sensation intensity when applied in the sensory feedback training and the evaluation tests. By interleaving the order of designated electrode pads, or by making the determination of threshold levels more scheme related (setting threshold levels simultaneously for pads connected in the schemes), could have made the sensation intensities more homogeneous across all pads. A weak functional electrical stimulation due to summation of active stimulation pads was observed in few subjects during the amplitude sensory feedback training, and might also have been avoided by relating the determination of threshold levels to the schemes. 

%Subject maxing out the amplitude during determination of thresholds.
\subsection{Future Works}
As mentioned, even with a minimal training, the results indicated a clear intuitiveness in understanding both feedback schemes. In that context, it would of great interest to investigate the performance of the feedback scheme concepts in a less discretized environment. With the electrode array used in this study, especially the amplitude scheme has a huge potential, as only the device restrictions and subjects' sensory discrimination abilities are a limit.
In the evaluation tests, only the active movement of the grasping DoF (closing the hand) was assessed, and the starting point was always resting state (no feedback received). In future studies, it could be investigated how the performance would be if the starting point was varied, e.g. by randomizing the starting point to resting state and highest level hand aperture, or by not resetting the prosthetic state when a new target state appeared. This would demand the subjects to more comprehensive understand the feedback, as they would not as rigidly be given reference states during the tests, and the test would be more transferable to practical prosthetic use.

Finally, as the schemes were easy comprehensible, an expansion of the scheme concepts to represent more DoF's would be a large step towards producing a prosthetic device concept with the potential of enhancing the users' prosthetic embodiment. Since electrotactile stimulation allows for modulation of frequency, another DoF could be included enhancing the complexity and amount of information which can be conveyed. For instance, proportional grasp force feedback was the most important feedback to restore according to \cite{Peerdeman2011}. This could be restored using frequency in a similar fashion as done in \cite{Dosen2016}, where grasp force was modulated via stimulation frequency. 
 
%Make comparison between amplitude and frequency modulated feedback. 



