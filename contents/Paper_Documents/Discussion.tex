Two intuitive electrotactile feedback schemes were developed for a two DoF velocity-based virtual prosthesis: one spatially modulated and one amplitude modulated. The schemes were integrated in an easy implementable 16 pad electrode array and tested in combination with sequential proportional myoelectric control. Unique sensory feedback was provided for four levels of prosthetic states in single DoF's and for 16 prosthetic states representing combined DoF's. The objective was to investigate the usability of the developed feedback schemes when removing visual dependency.
From the metrics extracted describing the subjects' performance in the evaluation test, only completion rate indicated a slight  dominance in favour of the amplitude scheme compared to the spatial scheme (p-value = 0.044). However, with a mean completion rate of 93 \% and 87 \%, respectively, both feedback schemes can be deemed intuitive to utilize in combination with myoelectric control when removing visual dependency. Considering that these completion rates were obtained from a minimal training protocol (training time per scheme < 30 minutes), a completion rate similar to visual feedback (99 \%) might be achieved if more training blocks were included. 


