%\newpage
\section{Short Experiment Description} \label{SED}

\textbf{Project Title} \\
Evaluation of Electrotactile Feedback Schemes in
Combination with Myoelectric Control.

\textbf{Experiment Purpose} \\
When a person gets amputated on the lower arm, he/she can receive a functional prosthesis. This is controlled by muscle signals from the user, where the muscle signals are translated into a prosthesis movement. However, many functional prostheses do not provide sensory feedback, which results in some users to abandon their prosthesis.\\
The purpose of the experiment is to compare how subjects perform in an evaluation test when receiving feedback from two different electrical stimulation configurations, while controlling a virtual prosthesis. The results might provide information on which feedback that seems more intuitive to use in a real prosthesis.   

\textbf{Experiment Overview} \\
The experiment will take place in the laboratory D3-107 at Aalborg University. The duration of the experiment is estimated to be 2 hours and 30 minutes. \\
During the experiment a myoelectric armband will be placed on the dominant forearm and used to record muscle activity while the subject performs four different hand gestures. Subsequently, an evaluation of the ability to reproduce the gestures will be made. \\
Afterwards, an electrode armband capable of delivering electrical stimulation at 16 different locations will be placed on the non-dominant arm. The subject will determine the level of sensory perception and tolerance level. The subject will then be made familiar with and trained in understanding two different feedback configurations that represents the possible states the virtual prosthesis can be in. At the end, an evaluation of the subject’s ability to understand the feedback while making the trained hand gestures will be made. \\
At the day of the experiment, please refrain from using any types of sensory deprivation drugs (painkillers and alike). The test subject will not receive monetary compensation after the experiment. 
