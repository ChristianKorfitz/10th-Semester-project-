\newpage
\section{Experiment Introduction Letter}

\textbf{Project Title} \\
Evaluation of electrotactile feedback schemes in combination with myoelectric prosthetic control - closing the loop. 

\textbf{Experiment Purpose} \\
The purpose of the experiment is to compare how subjects' perform in an evaluation test when receiving feedback from two different electrotactile stimulation configurations, while controlling a virtual prosthesis. The results might provide information on which feedback that seems more intuitive to use in a real prosthesis.   

\textbf{Experiment Overview} \\
The experiment will take place in the laboratory building $D3-107$ at Aalborg University. The duration of the experiment is approximated to 2 hours and 30 minutes. During the experiment a myo-electric armband will be placed on the dominant forearm and will be used to record the muscle activity during the performance of four different  hand gestures. Subsequently, a test of the ability to reproduce the gestures will be made. \\
Afterwards, an electrode, capable of delivering electrical stimulation will be placed on the non-dominant arm. A test to determine the electrical perception and tolerance level for the subject will then be carried out. The subject will then be made familiar and trained in understanding two different feedback configurations representing possible states which the virtual prosthesis might produce. A test of the subjects ability understand the feedback while making hand gestures will be made right after familiarization and training for each feedback configuration. \\
On the day of the experiment please refrain from using any types of sensory deprivation drugs (painkillers and the likes). Any form of reimbursement will not be provided at the end of the experiment.   