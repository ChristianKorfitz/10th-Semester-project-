\section{Experiment Protocol}

\textbf{Project Title}
Evaluation of electrotactile feedback schemes in combination with myoelectric prosthetic control - closing the loop. 

\textbf{Information on Investigators}
The investigators are biomedical engineering Master students at Aalborg University. 

\textbf{Background}
Losing an upper limb can be hugely debilitating and can result in lowered quality of life due to restrictions in function, appearance and sensation. As a mean to regain the loss, transradial amputees can receive a functional prosthesis, where the majority are controlled by muscles signals, or electromyography (EMG). However, still 25\% of EMG prosthesis users reject their device, where a major reason for the low satisfaction is due to lack of sensory feedback.
Many advancements have been made in the academic community to improve function accuracy. However, combining function with sensory feedback, thus closing the motor/sensory loop, is still a scarcely investigated area. Therefore, this experiment will combine the control of a prosthesis with sensory feedback delivered via electrotactile stimulation electrodes placed on the forearm. During the experiment the subjects will test two different feedback configurations while controlling a virtual prosthesis, represented as a cursor on a computer screen.  

\textbf{Purpose}
The purpose of the experiment is to compare how subjects' perform in an evaluation test when receiving feedback from two different electrotactile stimulation configurations, respectively, in a closed loop virtual prosthesis. This might provide information on which feedback that seems more intuitive to use in practice in a prosthesis.


\textbf{Research Aim}
Test and evaluate two novel stimulation schemes, one based on modulating amplitude and one based on spatial localization of activation, for conveying sensory feedback of the prosthesis state in a closed loop prosthetic control system.

\textbf{Experiment Duration}
Who knows

\textbf{Inclusion Criteria}
The subject must be:
\begin{itemize}
	\item able bodied or transradially amputated as the highest degree of upper limb amputation.
	\item at least 18 years of age.
	\item able to understand, read and speak English and/or Danish.
	\item assessed by the investigators to comply with the instructions given during the experiment.
\end{itemize}

\textbf{Exclusion Criteria}
The subject must:
\begin{itemize}
	\item not have any diseases/conditions that may influence sensory perception.
	\item be willing to receive low amplitude current stimulation. 
\end{itemize}


\textbf{{\Large Experiment Procedure}}

\textbf{{\Large Wrist Movements Used in the Experiment}}

\textbf{{\Large Experiment Setup}}

