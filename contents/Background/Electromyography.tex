\section{Electromyography}

The control of a myoelectric prosthesis is based on recorded myoelectric signals. \cite{Geethanjali2016}  Enabling the use of myoelectric signals for control of functional prosthetics requires a theoretical background knowledge of the signals origin and how it can be acquired. The following section will describe myoelectric signals and how they are acquired through the acquisition method of electromyography (EMG).      
 
The process of executing a voluntary movement can be explained through electric potentials and the excitability of skeletal muscle fibers. The nerve impulse carrying excitation information of a voluntary muscle contraction will travel from the motor cortex down the spinal cord to a alpha motor neuron. The alpha motor neuron cell will activate and direct the nerve impulse along its axon to multiple motor endplates, which each innervate muscle fibers. \cite{Turker2013} This initiates the release of neurotransmitters forming an endplate potential. The muscle fibers consist of muscle cells, which each are surrounded by a semi-permeable membrane. The resting potential over the membrane is held at a equilibrium, typically -80 mV to -90 mV, by ion pumps, which passively and actively control the flow of ion through the membrane. The release of neurotransmitters affects the flow through the ion pumps resulting in a greater influx of Na$^+$. This results in a depolarization of the cell membrane. However, only if the influx of Na$^+$ is great enough to create a depolarization surpassing a certain threshold, an action potential is formed. The action potential is characterized by the cell membrane potential, which changes from around -80 mV to +30 mV. After the depolarization a repolarization phase occurs and is followed by a hyperpolatization period, restoring the resting potential. The created action potential will propagate in both directions on the surface of the muscle fiber. The summation of this process and all its antagonist is in summation one motor unit. Hence, the action potential is also known as a motor unit action potential (MUAP), and it is the superposition of these across the muscle fibers, that is recorded through EMG. \cite{Turker2013,Martini2012} \\
Acquisition of EMG-signal can either be carried out through surface EMG or intramuscular EMG. The latter measures the MAUPs through needles inserted into the muscle and the former through electrodes on the skin surface. \cite{Cram2012} Using surface EMG requires preparation of the skin surface to minimize impedance and maximize skin contact. Hence, the skin should be clean and dry before electrode placement. Often considered is removing excess body hair or flaky skin and cleansing the area using alcohol swabs. \cite{Turker2013,Cram2012}  

%surface and iemg emg             
