\section{State of Art in Electrotactile Feedback} \label{SoA}

\Secref{SFS} presented different types of sensory feedback from which the choice of stimulation in this project can be drawn upon. Somatotopical feedback might provide the most natural sensations, but is also the most complicated to implement. Modality matching the feedback should instead be sought, however present tactors are larger and more power consuming than electrodes uses in electrotactile feedback. Furthermore, the dimensions of stimulation electrodes facilitate easier integration with the prosthesis as these can be placed inside the socket, along with electrodes used for acquisition. However, this requires that a solution for leakage current is found. Modulating pulse width, frequency and amplitude in electrotactile feedback gives more possibilities for conveying complex tactile information. Therefore, the state of art methods using electrotactile sensory feedback in the current literature have been reviewed and will presented to ensure that the later derived feedback schemes extends previous investigations. \\
Multiple studies have investigated the use of electrotactile feedback regarding both how distinguishable sensations can be evoked and how to convey sensory feedback in different coding schemes for improving myoelectric prosthetic control \cite{Stephens-Fripp2018}. 
In 2015, Shi and Shen \cite{Shi2015} investigated how subjects would perceive the effects of varying amplitude, frequency and pulse width of an electrical stimulation in various combinations. Results showed that appropriate sensations from electrical stimulation would be achieved by varying amplitude from 0.3 to 3 mA, pulse width from 0.1 to 20 ms and frequency from 40 to 70 Hz. Furthermore, varying these ranges properly would make it possible to have proportionally increased stimulation grades felt by the subject. Additionally, the authors stated the importance of electrode size, as stimulation through too large or too small electrode diameters could result in sensations of pain or discomfort. \cite{Shi2015} \\         
Several studies \cite{Pamungkas2015,Xu2016,Jorgovanovic2014,Isakovic2016} using electrical stimulation have investigated its use in conveying grasping force/pressure feedback. Jorgovanovic et al. \cite{Jorgovanovic2014} investigated users' recognition of grip strength, when controlling a joystick controlled robotic hand, through varying the pulse width and keeping the frequency and amplitude constant at 100 Hz and 3 mA, respectively. Results showed that providing electrotactile feedback improved the users' ability to move objects with the robotic hand. \cite{Jorgovanovic2014} Similar result were found by Isakovic et al. \cite{Isakovic2016}, who also showed that electrotactile feedback supported a faster learning than no feedback in grasp force control, and that electrotactile feedback might facilitate short-term learning. \\ 
A study by Xu et al. \cite{Xu2016} tested and evaluated different types of pressure and slip information feedback through electrotactile stimulation and compared this to visual feedback and no feedback. Electrotactile feedback was provided by keeping the intensity and frequency constant and then varying the pulse width between 0 and 500 $\mu $s indicating changes in grasp force. In this case, visual feedback was found to outperform electrotactile feedback. \cite{Xu2016} \\
Pamungkas et al. \cite{Pamungkas2015} also tested the use of electrotactile feedback to convey information from pressure sensors located in a robotic hand. Their setup used six feedback channels corresponding to a pressure sensor in each of the fingers and one in the palm. Pressure information in the sensors were given in three discretized frequency levels of 100, 60 and 30 Hz for the fingers and 20 Hz for the palm. Reported results stated that the subjects learned how to appropriately use the feedback when picking up objects of various sizes. Furthermore, the subjects reported that they preferred having electrotactile feedback accompanied by visual feedback opposed to only having visual feedback. \cite{Pamungkas2015} 
The purpose of restoring the sensation that would be experienced by touch of the skin has also been pursued in more elaborate efforts through artificial skin \cite{Hartmann2014,Franceschi2015}. In these cases, a grid of 64 pressure sensors were used to translate information of touch into 32 electrotactile electrodes placed on the arm of the subjects. \\
The use of electrotactile feedback has proven useful in cases of restoring the haptic feedback through pressure sensors on a prosthetic hand or by the touch on artificial skin. However, the possibilities of electrotactile feedback have also been investigated in the case of improving prosthetic control. In 2016, Strbac et al. \cite{Strbac2016} presented a novel electrotactile feedback stimulation system, which could be used to convey information about the current state of a multi-DoF prosthesis. The system comprised of four different dynamic stimulation patters communicating the states of four different DoF's through a 16 multi-pad array electrode, possibly restoring both proprioception and force feedback. The state of three of the DoF's were communicated by altering the electrodes activated in patterned fashion and the fourth DoF by modulating the stimulation frequency. Tests of the stimulation design showed that six amputees were able to recognize the four DoF's with an average accuracy of 86 \percent~while able-bodied subjects had a success rate of 99 \percent. \cite{Strbac2016}   \\
In summary, most studies have focused on using electrotactile feedback for exteroceptive means while only few have investigated its use for proprioceptive feedback. However, studies investigating proprioceptive feedback encourage further investigation into how electrotactile feedback can be utilized for providing meaningful proprioceptive feedback \cite{Strbac2016}.    

\subsection{Sensory Adaptation in Electrotactile Feedback}

Before implementing an electrotactile feedback interface, it is important to consider the effect electric stimulation might impose on the sensory system. \\
Adaption is defined as a changing sensory response to a constant stimulus, and all sensory systems have shown adaptive tendencies \cite{Buma2007}. This could result in undesired effects during prolonged electrical stimulation. Hence, it is crucial to consider stimulation parameters which reduce adaption. Sensory adaption usually occurs within minutes, and reaches a maximum after 15 min. Furthermore, the adaption rate is related to the stimulation amplitude as adaption occurs faster when closer to the pain threshold. Low frequencies (<10 Hz) show less adaption compared to higher frequencies (>1000 Hz). The adaption response is found to be exponential in decay and recovery. \cite{Buma2007,Szeto1982} 
However, sensory adaptation can be overcome by using intermittent stimulation, and preferably, stimulation interfaces should consider conveying feedback information through diversified patterns \cite{Szeto1982,Dosen2016}. \\
Developed feedback schemes should consider using as low amplitudes as possible to reduce the rate of sensory adaption. Furthermore, continuously changing the site of stimulation should also facilitate less adaption. 



