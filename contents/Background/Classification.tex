\section{Classification}
For the myoelectric prosthesis to know which movement to perform, it needs to know how to differentiate between the movements it will be trained to perform. For this purpose classification is a commonly applied model. The classification model, or classifier, is given known data consisting of features extracted from the raw EMG signals, which were recorded while the user was performing different movements. If each of the known feature data sets related to each movement is known they can be labelled appropriately, and the classifier will then know which data represents which movement. Each label is known as a class and the process of labelling the data is called supervised learning. The known data is also called training data, hence this process is called training the classifier. If the classifier is trained properly, it is able to categorize unknown data accurately into the correct class. This is what happens online when using a pattern recognition-based mypelectric prosthesis. The classifier is, however, only able to categorize unknown data into one of the trained classes. \\
A frequently used supervised classifier for myoelectric prosthetic control is the Linear Discriminant Analysis (LDA). An advantage of using LDA is that it enables robust control, while having a low computational cost. LDA will be used in this project as a control scheme and an overview of the theory behind LDA will be given in the following section.

\subsection{Linear Discriminant Analysis} 


fitcdiscr(classInput,moveLabel,'DiscrimType','pseudolinear', 'ScoreTransform','none','HyperparameterOptimizationOptions','bayesopt')

pseudolinear: all classes have the same covariance matrix. The matrix is inverted by the used the pseudo inverse

none: the confidence scores are not transformed

bayesopt: uses bayes optimization to yield minimum loss in cross validation