\section{Data Processing}
In order to use the acquired data most optimally in the myoelectric prosthetic control scheme the data must be processed. In this processing, undesired frequencies are filtered out and features that represents the data are extracted from segments of the data in order to obtain more information about the movement than what is only contained in the raw EMG signal. This data processing will be covered in the following sections. 

\subsection{Filtering}
To remove unwanted frequencies from the EMG signal, it is filtered. According to the Nyquist Theorem, the rate the signal is sampled with must be at least twice the highest frequency contained in the signal to archive a non-aliased digital recording. However, as mentioned in \secref{sec:MYO}, MYB samples with a rate lower than the highest frequency in the EMG spectrum, without having any analogue bandpass filter implemented. The rationale behind incorporating a digital anti-aliasing filter is therefore defeated. Implementing a digital high-pass filter to remove low frequency artefacts would, however, be desirable.  

\subsection{Feature Extraction}
Instead of only utilizing the raw EMG signal in a control scheme, features are extracted to exploit more representations of the EMG signal that optimally results in robust control. Various independent features can be extracted from the signal either from the time domain, frequency domain or the time-frequency domain. Most commonly features from the frequency and time domain are used. When extracting frequency domain features it is required for the EMG signal to transformed into the frequency domain. This takes more computation time compared to extracting features directly from the time domain. For this reason features in the time domain are usually favoured. \cite{Phiny2012} Especially used are the Hudgins features: Mean Absolute Value (MAV), Zero Crossings (ZC), Slope Sign Changes (SSC) and Waveform Length (WL) \cite{Hudgins1993}. However, both ZC and SSC represent the frequency content of the signal, which most likely has been distorted by the low sample rate. When using MYB for EMG acquisition an alternative set of features has been suggested by Donovan et al. to extract from the data \cite{Donovan2017}. These features are so called space domain feature, since they exploit the relationship between the output from the electrode channels. When evaluating data acquired from MYB the space domain features increased classification accuracy by 5 \% in a LDA-based control scheme compared to using the Hudgins feature \cite{Donovan2017}. A final consideration to make when choosing features is to avoid redundancy as they then would not provide additional information about the signal. 

\subsection{Segmentation}
The extraction of features are done in discretely segmented windows of data, instead of calculating the features from instantaneous values. In online control, the length of windows is a compromise between classification accuracy and delay in prosthetic control. Often an window overlap is implemented. This is a technique applied to ensure short delays, while still enabling a high classification accuracy. When applying an overlap values from the previous window is reused in the current window. The amount of overlap chosen is significant for the performance of the control scheme. Choosing a large overlap will result in short delays, but worse classification accuracy and vice versa. When using MYB it is important to take the low sample rate into consideration, as a window will contain less data compared to if the sampling was appropriate to the EMG frequency properties. Short windows will likely result in worse classification accuracy compared to appropriately sampled data segmented in identical window length. 
