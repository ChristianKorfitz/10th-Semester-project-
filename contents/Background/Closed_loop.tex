\section{Closing the Loop}

The loss a limp does not only result in a loss of motor function, sensory function is also impaired. Providing an amputee with a prosthetic device, which does not provide sensory feedback, only restores one half of the one closed limb control loop. To close the loop the prosthetic device needs to contain proprioceptive and exteroceptive sensors, which recorded information needs to be conveyed to the amputee in a intuitive and meaningful way \cite{Markovic2018}. This can be achieved using the before mentioned methods of sensory substitution \cite{Schweisfurth2016}. \\
Closing the loop is a well recognized need of prosthetic users and might improve easiness of use and embodiment, possibly lowering rejection rates. Furthermore, the need for visual attention to correct prosthetic movement would be lowered. \cite{Strbac2016} However, the advantages of closing the loop by providing sensory substitution feedback have been contradictory \cite{Jorgovanovic2014}. In 2008 Cipriani et al. \cite{Cipriani2008} investigated the use of vibroctacile feedback for improving grasp in a prosthetic and did not find any improvement using providing the sensory feedback. Later finding by Witteveen et al. \cite{Witteveen2012} disproved this as they found providing information of grasp force and slip through vibrotactile feedback improved a virtual grasping task. \\
Even though most studies find closing the loop by providing sensory feedback helpful (review by Stephens-Fripp et al.) \cite{Stephens-Fripp2018}, currently one device, the VINCENT evolution 2 (Vincent Systems Gmbh, DE) is commercially available conveying grasp force feedback \cite{Systems2005}.  
Addtionally, closed loop control systems bypassing human interaction have also been investigated and implemented by commercial manufacturers i.e. Otto Bock and RSL steeper. Actuators are made to autonomously adjust grip force based on sensor located in the prosthetic hand. \cite{Xu2016} Such an approach might improve reliability of the prosthesis, but does not provide proprioceptive and exteroceptive to the user thus not promoting embodiment.  




