\section{Closing the Loop}

The loss of a limp does not only result in loss of motor function as sensory function also gets impaired. Providing an amputee with a prosthetic device, which does not provide sensory feedback, only restores one half of the once closed limb control loop. To close the loop the prosthetic device needs to contain proprioceptive and exteroceptive sensors, whose recorded information should be conveyed to the amputee in a intuitive and meaningful way \cite{Markovic2018}. This can be achieved using methods of sensory substitution mentioned in \secref{senssub}. \\
Closing the loop is a well recognized need amongst prosthetic users and might improve easiness of use and embodiment, which might lower rejection rates. Furthermore, the need for visual attention to track correct prosthetic movement would be lowered. \cite{Strbac2016} However, the advantages of closing the loop by providing sensory substitution feedback have been contradictory \cite{Jorgovanovic2014}. In 2008, Cipriani et al. \cite{Cipriani2008} investigated the use of vibroctacile feedback for improving prosthesis grasp function and did not find any improvement when providing the sensory feedback. Later findings by Witteveen et al. \cite{Witteveen2012} disproved this as they found that when providing information of grasp force and slip through vibrotactile feedback improved a virtual grasping task. \\
Even though most studies find closing the loop by providing sensory feedback helpful (review by Stephens-Fripp et al.) \cite{Stephens-Fripp2018}, currently only one commercial feedback providing device, the VINCENT evolution 2 (Vincent Systems Gmbh, DE) is available \cite{Systems2005}.  
Addtionally, closed loop control systems bypassing human interaction have also been investigated and implemented by commercial manufacturers i.e. Otto Bock and RSL steeper. Actuators are made to autonomously adjust grip force based on sensors located in the prosthetic hand, thereby not involving the user in the final execution of the task. \cite{Xu2016} Such an approach might improve reliability of the prosthesis, but does not provide proprioceptive and exteroceptive feedback to the user, hence not promoting embodiment.  




