The background chapter will outline the considerations that need to be made when testing the usability of two different sensory feedback configurations in connection to myoelectric prosthetic control. The feedback will be given based on which motion state the prosthesis is in, which is based on the output from a pattern recognition control system. \\
The main idea behind myoelectric prosthetic control is to translate recorded muscle signals (EMG signals) into a motion performed by the prosthesis. The pattern recognition model has been taught to differentiate between a set of movement classes. When receiving a time portion of a EMG signal it then decides upon which movement class that most likely is being performed. In combination with the muscle contraction level, this is used as output to the control system and the prosthesis performs a motion. In a closed loop prosthesis, the motion state the prosthesis is in will be equivalent to a certain sensory feedback. The user is then able to interpret the sensory feedback and use as additional information to visual feedback about the prosthesis' state. A closed loop prosthesis iteration can be seen in \figref{fig:closed_loop_pros}. \\
Regarding control the background chapter will explain the following: generation of EMG signals, data acquisition, data processing, pattern recognition and proportional control. Regarding sensory feedback the following will be explained: prior investigations on sensory feedback, types of sensory feedback and sensory feedback configurations. 

\begin{figure}[H]                 
	\includegraphics[width=.4\textwidth]{figures/closed_loop_pros}  
	\caption{The figure shows the stages of a closed loop prosthesis. First, EMG signals are recorded from the user. The signals are decoded and an output is relayed to the control system, which is used for the prosthesis to perform a motion. The motion state is then read and sensory feedback is delivered to the user regarding which motion state the prosthesis is in.}
	\label{fig:closed_loop_pros} 
\end{figure}