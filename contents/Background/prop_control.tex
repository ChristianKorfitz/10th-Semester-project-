\section{Proportional Control}
After the motor function has been determined, a mapping of a control output needs to be performed. The advantage of providing a continuous output to the actuator proportional to the contraction intensity compared to a one-speed controller is that the user has the possibility of grasping objects quickly, while still being able to perform more slow and dexterous tasks. Additionally, proportional control resembles the human neuromotor system, which makes it more intuitive. \cite{Fougner2012} \\
A widely used proportional control scheme is linear regression \cite{Fougner2012}. Here, a dependent output value can be calculated based on a function of an independent input value. In the case of using several electrode channels as when using the MYB, the output needs to be computed based on several independent values. For this purpose multivariate linear regression would be appropriate:

\begin{equation}
	\hat{Y} = \alpha+\beta_{1}X_{1}+\beta_{2}X_{2}+\cdots+\beta_{i}X_{i}+\epsilon_{i}
\end{equation}

where $\hat{Y}$ is the control output and $X_{i}$ is the independent input values, where the index $i$ will correspond to the number of electrode channels in the MYB. $\alpha$ and $\beta$ are the estimated value of $\hat{Y}$ at $X$ = 0 and estimated regression coefficients, respectively. The absolute values of the recorded EMG signals can be used directly as the independents input value in such a proportional control scheme. \cite{Zar2009} However, a regression model needs to be estimated for each motor function in the control system. Then the appropriate regression model will be selected based on the classification output.  