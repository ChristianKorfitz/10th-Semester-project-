\section{Performance Evaluation}

Evaluating the performance of a derived control system can be achieved through the completion of various tasks. If available, the system can be interfaced with a myoelectric prosthesis, and based on the completion of tasks mimicking daily life functionality (e.g. grasp and movement of objects), performance can be evaluated \cite{Mastinu2018}. Otherwise, virtual environments have been widely used showing movements of virtual prostheses \cite{Powell2014} or by moving a cursor to targets resembling motor function, where performance can be quantified through measurements based on Fitts' Law \cite{Scheme2013,Wurth2014,Hahne2014}. An obvious measure to observe is the completion rate (CR), which is the ratio of reached targets compared to the total number of targets. This describes the overall ability the user has when using the control system. Path efficiency (PE) can be used to observe how efficiently continuous movement control is achieved by comparing the distance travelled to reach a target and comparing it to the most direct route. To observe how well the user can keep the system at rest and control velocity, stooping distance (SD) and overshoot (OS) can be measured. The former measures the distance travelled at times where no movement is intended, and the latter tracks the number of times the user reaches a shown target, but leaves before completion. \cite{Scheme2013}
