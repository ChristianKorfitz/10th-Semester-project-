\section{Performance Evaluation}

Evaluating the performance of a derived control system can be achieved through the completion of various tasks. If available, the system can be interfaced with a myoelectric prosthesis and based on the completion of tasks mimicking daily life functionality (e.g. grasp and movement of objects) performance can be evaluated \cite{Mastinu2018}. Otherwise, virtual environments have been widely used showing movements of virtual prostheses \cite{Powell2014} or by moving a cursor to targets resembling motor function measuring metrics based on Fitts's Law \cite{Wurth2014,Scheme2013,Hahne2014}. In this project performance evaluation will be carried out in a virtual environment through a Fitts' Law based target reaching test. The next section will present metrics representing aspects of performance. 

\subsection{Fitts' Law Test}        

Various versions of the Fitts' Law test focusing on different performance metrics have been derived to quantify performance of myoelectric control. The metrics are used to describe different aspect of completing a movement task. \\
When designing the test, the Index of Difficulty (ID) if often calculated for each target in order to asses how difficult it is to reach a corresponding target. The ID is based on the distance to the target and target width.  
Most obvious to observe is the completion rate (CR) which is the ratio of reached targets compared to all targets. This describes the overall ability the user has when using the control system. \\
Throughput (TP) describes the achieved speed and accuracy by using the relationship of time used to reach a target compared to the target ID. 
Path efficiency (PE) can be used to observe how efficiently continuous movement control is achieved by comparing the distance travelled to reach a target and comparing it to the most direct way. 
To observe how well the user can keep the system at rest and control velocity, stooping distance (SD) and overshoot (OS) can be measured. The former measures the distance travelled at times where no movement is intended while the latter tracks the number times the user reaches desired target but leaves before completion. \cite{Scheme2013}
