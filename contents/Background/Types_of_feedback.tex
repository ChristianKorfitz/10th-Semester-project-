\section{Sensory Feedback Stimulation} \label{SFS}

It has been known for some time that vision alone does not provide a sufficient amount information to achieve efficient daily life use of a prosthetic device, as the use requires full visual attention attention. Hence, efforts have been put in investigating methods of providing proprioceptive and exteroceptive information of i.e. grasp strength and prosthetic state through the means of artificial stimulation. \cite{Schofield2014,Stephens-Fripp2018} Presently, there are multiple ways of providing the user with a variety of sensory feedback. These can be divided into three categories: Somatotopically feedback, modality matched feedback and substitution feedback. \cite{Schofield2014} \\
This section will present general terms in sensory feedback stimulation and give a brief overview of the types of sensory feedback in order to give insight in the possibilities and eventual disadvantages when providing the user of a prosthetic device with feedback.

\subsection{Somatotopically Feedback}

Somatotopically feedback aims to provide the user with a sensory experience which is perceived as natural as what was felt by their missing limb, both in location and sensation. To achieve such an experience, somatotopically feedback uses invasive approaches by making use of invasive neural electrodes and targeted reinnervation. The former is known as peripheral nerve stimulation and relies on the invasive neural electrodes being interfaced with the original neural pathways preserved proximally on the residual limb. Currently, two different types of electrodes have been exploited: One where a cuff is placed around a nerve fascicle and another where an electrode is implanted into the nerve fiber. But to this date, none of these methods have been comprehensively studied. Targeted reinnervation also enable the possibility of stimulating the original neural pathways from the missing limb. The corresponding sensory afferents are relocated to innervate new sites which can selectively be chosen and stimulated by non-invasive tactors. Somatotopically matched feedback is hypothesized to reduce the users cognitive burden due to its 'naturalness', facilitating increased compliance and less conscience attention. \cite{Schofield2014}  

\subsection{Modality Matched Feedback}

In modality matched feedback, the type of sensory experience which would have been felt by the missing limb is communicated to the user. For instance, when pressure is felt in the palm of a prosthetic hand by pressure sensors, a proportional amount of pressure is delivered to the user somewhere on the skin. Thus, the sensation is not matched in location, but only in sensation. Mechanotactile feedback which conveys pressure information is utilized by the use of i.e. pressure cuffs or servomotors. These types of tactors are very useful for modality matched feedback, but have a disadvantage by being more power consuming compared to other stimulation types. \cite{Schofield2014,Antfolk2018} 

\subsection{Substitution Feedback} \label{senssub}

Substitution feedback methods convey information about the state of the prosthesis without regarding the type of sensation and location which would have been felt by the missing limb. Thereby, the sensory information is said to be non-physiologically representative. The feedback methods are often straightforward to implement, but demands a greater amount of user adaption to interpret what the feedback information represents. Often used methods for substitution feedback are vibrotactile and electrotactile feedback. \cite{Schofield2014,Antfolk2018}     

\subsubsection{Vibrotactile Stimulation}

Vibrotactile stimulation utilizes small mechanical vibrators to convey information to a selected area of the skin which activates cutaneous mechanoreceptors. This method is most often used to transfer tactile information in prosthetic grasping tasks. \cite{Schofield2014} A recognizable sensation is evoked using frequencies between 10 and 500 Hz. The sensory threshold varies between users and location, resulting in the need for specific user threshold calibration. \cite{Antfolk2018}  


\subsubsection{Electrotactile Stimulation} \label{E-stim}

In electrotactile feedback a sensory sensation is achieved by stimulating the primary myelinated afferent nerves with an electrical current. This creates what is often referred to as a tingling sensation. Electrotactile stimulation rely on small and lightweight electrodes to provide the electrical stimulation. When compared to other feedback methods as vibrational and pressure stimulation, which depend on heavier actuators and moving parts to provide the feedback, these properties can be seen as a drawback as prosthetic users strongly desire lightweight systems \cite{Stephens-Fripp2018,Benz2016}. Furthermore, through the use of electrotactile stimulation, multiple factors such as amplitude, pulse width, frequency and location of the stimulation can be controlled facilitating development of agile feedback schemes. This enables the possibility of varying the perceived feedback as either vibration, tapping or touch by modulating the signal waveform. The downside of using electrodes is the requirement for recalibration of sensory thresholds, pulse width and frequency to reproduce the same perceived stimulation every time the electrodes are placed on the user. In addition, interference between electrodes used for stimulation and recording have been found to result in noise in recorded EMG-signal used for myoelectric control. Concentric electrodes are able to limit the interference by limiting the spread of current. Concentric electrodes have also been found to increase localization and perceptibility of the induced stimuli. \cite{Schofield2014,Stephens-Fripp2018,Antfolk2018} 







%Prosthetic users have also shown a strong desire to decrease the need for visual attention to perform functions