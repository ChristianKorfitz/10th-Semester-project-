\section*{Resume}

Tabet på en overekstremitet kan være yderst traumatiserende og livsændrende og lede til signifikant nedsat livskvalitet grundet formindsket funktionalitet, sensation og udseende. Et aktiv med henblik på at genskabe den manglende funktionalitet hos trans-radialt amputerede findes i myoelektriske proteser. På trods af stor udvikling inden for proteseteknologier vælger 25 \% af protesebrugere alligevel at undlade brugen af myoelektriske proteser. En rapporteret årsag til dette er, at de tilbudte kommercielle proteser mangler proprioceptiv og exteroceptiv feedback. Protesebrugerne er derfor bundet af visuelt feedback for at afgøre protesens position i rummet, og mangler taktil information, om hvordan protesens interaktion med omgivelserne føles. Grundet denne mangel opfatter brugerne derfor ikke protesen som en integreret del af kroppen. 

Tidligere studier har forsøgt at genskabe disse sensationer, hvor vigtige aspekter dog er blevet negligeret. Manglen i disse studier omfatter simpliciteten af den simulerede feedback, den praktiske integration af stimulatortypen i en protese og manglende evaluering af feedback i kombination med myoelektrisk protesekontrol. For at imødegå disse aspekter blev der i dette studie evalueret anvendeligheden af to innovative stimulationskonfigurationer i en simuleret virtuel protese, der formidlede proprioceptiv feedback vedrørende to frihedsgrader (rotation af hånden og lukning af hånd) igennem et let integrerbart elektrotaktil feedback interface. De to frihedsgrader blev inddelt i fem diskrete intervaller, med en unik taktil feedback tilskrevet hvert interval. Da elektrotaktil feedback er multimodalt, var et yderligere fokus at evaluere hvilken modalitet, der kunne formidle taktil proprioception bedst muligt. De to modaliteter, der blev undersøgt, var spatial aktivering af forskellige områder i et elektrode array, og amplitude modellering af stimuleringen fra elektrode arrayet. 

14 testpersoner blev rekrutteret og evalueret gennem en Fitts’ Law inspireret test, som følge af en omtrent halv times lang træningssession i en af feedbackkonfigurationerne. Dernæst blev den anden feedbackkonfiguration trænet og evalueret. Hvilken konfiguration testpersonerne blev evalueret i først, var randomiseret. Indledningsvist blev testpersonerne trænet i at styre den virtuelle protese, da det var påkrævet at testpersonerne opnåede robust kontrol, før konfigurationerne kunne blive retfærdigt evalueret i en lukket motor/sensor loop protese.

Som resultat af evalueringstesten opnåede de amplitude og spatialt-baserede konfigurationer succesrater på henholdsvis 93 \% $\pm$ 6 \% og 87 \% $\pm$ 11 \%. Amplitude feedback konfigurationen opnåede en smule højere succesrate end den spatialt-baserede (p = 0.044), hvilket også var understøttet af testpersonernes subjektive vurdering, da 64 \% favoriserede amplitude feedbacken. Succesraten for begge konfigurationer var dog betydeligt lavere end for visuelt feedback (99 \% $\pm$ 2 \%, p < 0.001). Synet er imidlertid en mere dominerende sans inden for motorisk læring end proprioception, og en identisk succesrate ville derfor aldrig være forventeligt. 

Med så højt opnåede succesrater som følge af minimal træning, kan begge feedbackkonfigurationer anses som værende intuitive og let forståelige. Eftersom stimuleringssetuppet krævede minimal plads, vil det kunne integreres i en reel protese, hvilket potentielt kunne medføre en myoelektrisk protese, som brugere ville legemliggøre. Derudover havde især den amplitude-baserede feedbackkonfiguration potentialet til nemt at kunne udvides til at formidle endnu mindre diskretiserede feedbackintervaller, hvilket kunne yderligere udbygge anvendeligheden.
